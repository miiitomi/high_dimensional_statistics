\documentclass[aspectratio=169, dvipdfmx]{beamer}
\usetheme{Boadilla}
\usepackage{amsmath, amssymb, amsthm, color, latexsym}
\usefonttheme{professionalfonts}

\def\qed{\hfill $\Box$}
\newcommand{\ex}{\mathbb{E}}
\newcommand{\var}{\mathrm{var}}

\title{3. Concentration of measure}
\author{Yoji Tomita}
\date{May 12, 2021}

\begin{document}

\maketitle

% \begin{frame}{Table of Contents}
%     \tableofcontents
% \end{frame}

\begin{frame}{Introduction}
\begin{itemize}
    \item 2章を前提として, この章ではtail boundやconcentration inequalitiesを求めるためのより上級的な手法を紹介する.
    \item 3.1:Concentration by entropic techniques
    \item 3.2:A geometric perspective on concentration
    \item 3.3:Wasserstein distances and information inequalities
    \item 3.4:Tail bounds for empirical processes
\end{itemize}
\end{frame}

\section{3.1 Concentration by entropic techniques}

\begin{frame}{3.1 Concentration by entropic techniques}
\begin{itemize}
    \item エントロピーと, 集中不等式導出のためのその関連テクニックに関する議論から始める.
\end{itemize}
\end{frame}

\subsection{3.1.1 Entropy and its properties}
\begin{frame}{3.1.1 Entropy and its properties}
\begin{itemize}
    \item 凸関数 $\phi:\mathbb{R} \to \mathbb{R}$と, 確率変数$X\sim \mathbb{P}$に対して, $\phi$-entropyを
    \[\mathbb{H}_\phi(X) := \ex[\phi(X)] - \phi(\ex[X])\]
    とする($X, \phi(X)$の有限期待値は仮定).\\
     
    \item Jensenの不等式より, $\phi$-entropyは非負.
    \item これは$X$のばらつき加減を表す.
    \begin{itemize}
        \item 極端な場合, $X$がa.s.で期待値と一致するなら, $\mathbb{H}_\phi(X) = 0$.
    \end{itemize}
\end{itemize}
\end{frame}

\begin{frame}
\begin{itemize}
    \item 例1:$\phi(u) = u^2$なら$\mathbb{H}_\phi(X)$は分散.
        \[ \mathbb{H}_\phi(X) = \ex[X^2] - (\ex[X])^2 = \var(X).\]
     
    \item 例2:$\phi(u) = -\log u$, $Z := e^{\lambda X}$とすると,
        \[ \mathbb{H}_\phi(e^{\lambda X})  = -\lambda \ex[X] + \log \ex[e^{\lambda X}] = \log\ex[e^{\lambda(X-\ex[X])}]\]
        となり, centerd cumulant generating functionとなる.
\end{itemize}
\end{frame}

\begin{frame}
    \begin{itemize}
        \item この章では, 次の凸関数$\phi:[0,\infty)\to\mathbb{R}$に対するentropyを考える.
        \[\phi(u) := 
            \begin{cases}
                u\log u & \mathrm{for} \ u > 0\\
                0       & \mathrm{for} \ u = 0
            \end{cases}.\tag{3.1}\label{3.1}
        \]
        非負確率変数$Z$に対して, $\phi$-entropyは
        \[ \mathbb{H}(Z) = \ex[Z\log Z] - \ex[Z]\log\ex[Z], \tag{3.2}\label{3.2}\]
        となる(ただし関連する期待値の存在は仮定). 
        \begin{itemize}
            \item Shannon entropyやKullback-Leibler divergenceと関連がある(see Exercise 3.1).
            \item 以後このentropyを考えるので, $\mathbb{H}_\phi$のsubscrript $\phi$は省略.
        \end{itemize}
         
        \item $Z = e^{\lambda X}$とすると, $\mathbb{H}(e^{\lambda X})$は$X$のモーメント母関数$\varphi_X(\lambda)=\ex[e^{\lambda X}]$とその導関数$\phi_X'(\lambda)$で表せる.
        \[\mathbb{H}(e^{\lambda X}) = \lambda \varphi_X'(\lambda) - \varphi_X(\lambda)\log\varphi_X(\lambda). \tag{3.3}\label{3.3}\]
    \end{itemize}
\end{frame}

\begin{frame}
\begin{exampleblock}{Example 3.1 (Entropy of a Gauusian random variable)}
    \begin{itemize}
        \item $X$は1次元正規分布$X\sim \mathcal{N}(0, \sigma^2)$とすると, $\varphi_X(\lambda) = e^{\lambda^2\sigma^2/2}, \varphi_X'(\lambda) = \lambda \sigma^2\varphi_X(\lambda)$なので,
        \[ \mathbb{H}(E^{\lambda X}) = \lambda^2\sigma^2\varphi_X(\lambda) - \frac{1}{2}\lambda^2\sigma^2\varphi_X(\lambda) = \frac{1}{2}\lambda^2\sigma^2\varphi_X(\lambda). \tag{3.4}\label{3.4}\]
    \end{itemize}
\end{exampleblock}
 \\
\begin{itemize}
    \item この節の残りで, このエントロピー(\ref{3.3})とtail boundsとの関連性を説明していく.
\end{itemize}
\end{frame}

\subsection{3.1.2 Herbst argument and its extensions}
\begin{frame}{3.1.2 Herbst argument and its extensions}
\begin{itemize}
    \item ある定数$\sigma > 0$が存在して, エントロピーが次の上限を満たすとする.
    \[ \mathbb{H}(e^{\lambda X}) \le \frac{1}{2}\sigma^2 \lambda^2\varphi_X(\lambda).\tag{3.5}\label{3.5} \]
    \begin{itemize}
        \item Example 3.1より, 正規分布$X\sim \mathcal{N}(0,\sigma^2)$は任意の$\lambda \in \mathbb{R}$に対し(3.5)をイコールで満たす.
        \item また任意のboundedな確率変数も(\ref{3.5})を満たす(Exercise 3.7).
    \end{itemize}
    \item このとき, その確率変数はsub-Gaussianとなる.
\end{itemize}
\begin{block}{Proposition 3.2 (Herbst argument)}
    エントロピー$\mathbb{H}(e^{\lambda X})$が(\ref{3.5})を任意の$\lambda \in I$ (ただし$I=[0, \infty)\ \mathrm{or}\ \mathbb{R}$)について満たすとする.
    このとき,
    \[ \log \ex[e^{\lambda(X-\ex[X])}] \le \frac{1}{2}\lambda^2\sigma^2 \ \ \ \mathrm{for\ all}\ \lambda\in I. \tag{3.6}\label{3.6} \]
\end{block}
\end{frame}

\begin{frame}{}
Remarks:
\begin{itemize}
    \item $I = \mathbb{R}$なら, (\ref{3.6})は$X-\ex[X]$がパラメータ$\sigma$のsub-Gauusianであることと同値.
    \item Chernoff argumentより, $I = [0, \infty)$でも片側tail-bound
    \[ \mathbb{P}[X \ge \ex[X]+t] \le e^{-\frac{t^2}{2\sigma^2}} \tag{3.7}\label{3.7}\]
    が得られ, $I=\mathbb{R}$なら両側tail bounds
    \[ \mathbb{P}[|X-\ex[X]|\ge t] \le 2e^{-\frac{t^2}{2\sigma^2}}\]
    となる.
\end{itemize}
\end{frame}

\begin{frame}{}{} 
{\bf Proof.}
\begin{itemize}
    \item $I = [0,\infty)$の場合のみ示す($I=\mathbb{R}$は演習とする).
    \item エントロピーのモーメント母関数による表現(\ref{3.3})と仮定(\ref{3.5})より,
    \[\mathbb{H}(e^{\lambda X}) = \lambda \varphi'(\lambda) - \varphi(\lambda)\log\varphi(\lambda)\le\frac{1}{2}\sigma^2\lambda^2\varphi(\lambda) \ \ \ \forall \lambda \ge 0.\tag{3.8}\label{3.8} \]
    \item 関数$G$を$G(\lambda):= \frac{1}{\lambda}\log\varphi(\lambda)\ (\lambda\ne 0)$と定義し, $\lambda=0$では連続性を満たすように
    \[ G(0):= \lim_{\lambda \to 0}G(\lambda) = \ex[X] \tag{3.9}\label{3.9}\]
    とする.
    \item $G'(\lambda) = \frac{1}{\lambda}\frac{\varphi'(\lambda)}{\varphi(\lambda)}-\frac{1}{\lambda^2}\log\varphi(\lambda)$より, (\ref{3.8})は$G'(\lambda) \le \frac{1}{2}\sigma^2$となるので, $\lambda_0(>0)$から$\lambda$まで両辺積分すると
    \[ G(\lambda) - G(\lambda_0) \le \frac{1}{2}\sigma^2(\lambda-\lambda_0). \]
    \item $\lambda_0 \downarrow 0$とすると
    \[ G(\lambda) - \ex[X] \le \frac{1}{2}\sigma^2\lambda \]
    となり, これは(\ref{3.6})と同値である.\qed
\end{itemize}
\end{frame}

\end{document}