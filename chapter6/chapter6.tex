\documentclass[unicode,aspectratio=169,11pt]{beamer}
\usepackage{amsmath, amssymb, amsthm, color, latexsym, mathrsfs, bm}
\usefonttheme{professionalfonts}
\usepackage{luatexja}
\usepackage[ipaex]{luatexja-preset}
\renewcommand{\kanjifamilydefault}{\gtdefault}

\usetheme[
  sectionpage=none,
  numbering=fraction,
  block=fill
  ]{metropolis}

\def\qed{\hfill $\Box$}
\def\endexample{\hfill $\clubsuit$}
\newcommand{\ex}{\mathbb{E}}
\newcommand{\var}{\mathrm{var}}
\newcommand{\bb}{\mathbb}
\newcommand{\cc}{\mathcal}
\newcommand{\tr}{\mathrm{T}}
\newcommand{\trace}{\mathrm{tr}}


\title{6. Random matrices and covariance estimation}
\author{担当:みーとみ}
\date{2021年6月30日, 7月7日}


\begin{document}

\maketitle

\begin{frame}{Table of Contents}{}
    \tableofcontents
\end{frame}

\section{6.1 Some preliminaries}
\begin{frame}{6.1 Some preliminaries}
\end{frame}

\section{6.2 Wishart matrices and their behavior}
\begin{frame}{6.2 Wishart matrices and their behavior}
  \begin{itemize}
    \item サンプル $x_i$ は, $d$-次元正規分布 $\mathcal{N}(0, \Sigma)$ から i.i.d. で引かれるとする.
    \item このとき,
    \[
      X =
      \begin{pmatrix}
        x_1^\tr\\
        \vdots\\
        x_n^\tr
      \end{pmatrix} \in \bb{R}^{n \times d}
    \]
    は, {\it $\Sigma$-Gaussian ensemble} から引かれると言う.
    \item Sample covariance $\widehat{\Sigma} = \frac{1}{n}X^\tr X$ は, {a multivariate Wishart distribution} に従う.
  \end{itemize}
\end{frame}

\begin{frame}{}{}
  \begin{block}{Theorem 6.1}
    $X \in \mathbb{R}^{n \times d}$ は $\Sigma$-Gaussian ensembleから引かれるとする.
    このとき, 任意の $\delta > 0$ に対し, 最大特異値 $\sigma_{\max}(X)$ は以下のupper deviation inequalityを満たす:
    \[
      \bb{P} \left[
        \frac{\sigma_{\max}(X)}{\sqrt{n}}
        \ge \gamma_{\max}\left(\sqrt{\Sigma}\right) (1+\delta) + \sqrt{\frac{\trace(\Sigma)}{n}}
      \right]
      \le \exp\left(-\frac{n\delta^2}{2}\right).
      \tag{6.8}
    \]
    さらに$n \ge d$なら, 最小特異値 $\sigma_{\min}(X)$ は以下のlower deviation inequalityを満たす:
    \[
        \bb{P} \left[
        \frac{\sigma_{\min}(X)}{\sqrt{n}}
        \le \gamma_{\min}\left(\sqrt{\Sigma}\right) (1-\delta) - \sqrt{\frac{\trace(\Sigma)}{n}}
      \right]
      \le \exp\left(-\frac{n\delta^2}{2}\right).
      \tag{6.9}
    \]
  \end{block}
\end{frame}

\begin{frame}{}{}
 \\
{\bf Example 6.2} (Operator norm bounds for the standard Gaussian ensemble)
\begin{itemize}
  \item $W \in \bb{R}^{n \times d}$ は各成分が$\cc{N} (0,1)$ i.i.d.で引かれるrandom matrixとする($\Sigma = I_d$).
  \item Thm 6.1より, $n \ge d$ なら, 確率 $1 - 2 \exp\left(-\frac{n\delta^2}{2}\right)$ 以上で
  \[\frac{\sigma_{\max}(W)}{\sqrt{n}} \le 1 + \delta + \sqrt{\frac{d}{n}}
  \quad \mathrm{and} \quad
  \frac{\sigma_{\min}(W)}{\sqrt{n}} \ge 1 - \delta - \sqrt{\frac{d}{n}}
  \tag{6.10}
  \]
  となる.
  \item よって, 同じ確率で
  \[ \Bigg|\Bigg|\Bigg| \frac{1}{n}W^\tr W - I_d \Bigg|\Bigg|\Bigg|_2 \le 2\epsilon + \epsilon^2,
  \quad \mathrm{where} \epsilon = \sqrt{\frac{d}{n}} + \delta.
  \tag{6.11} \]
  \item したがって, $d/n \to 0$ なら, sample covariance $\widehat{\Sigma} = \frac{1}{n}W^\tr W$ はidentiry matrix $I_d$ の一致推定量となる.\endexample
\end{itemize}
\end{frame}

\begin{frame}{}{}
 \\
{\bf Example 6.3} (Gaussian covariance estimation)
\begin{itemize}
  \item $X \in \bb{R}^{n \times d}$ は$\Sigma$-Gaussian ensembleからのrandom matrixとする.
  \item このとき $X = W \sqrt{\Sigma}$ と書ける($W \in \bb{R}^{n\times d}$はstandard Gaussian random matrix)ので,
          \[
              \Bigg|\Bigg|\Bigg| \frac{1}{n} X^\tr X - \Sigma \Bigg|\Bigg|\Bigg|_2
              = \Bigg|\Bigg|\Bigg|\sqrt{\Sigma}\left(\frac{1}{n}W^\tr W - I_d\right) \Bigg|\Bigg|\Bigg|_2
              \le |||\Sigma|||_2 \Bigg|\Bigg|\Bigg| \frac{1}{n} W^\tr W - I_d \Bigg|\Bigg|\Bigg|_2.
          \]
  \item したがって (6.11) より, 任意の $\delta > 0$ に対して確率 $1 - 2\exp\left(-\frac{n \delta^2}{2}\right)$ で
          \[
            \frac{||| \widehat{\Sigma} - \Sigma |||_2}{|||\Sigma|||_2}
            \le 2 \sqrt{\frac{d}{n}} + 2\delta + \left(\sqrt{\frac{d}{n}} + \delta\right)^2.
            \tag{6.12} 
          \]
  \item よって, $||| \widehat{\Sigma} - \Sigma |||_2 / |||\Sigma|||_2$ は $d/n \to 0$ である限り $0$ に収束する.\endexample
\end{itemize}
\end{frame}

\begin{frame}{}{}
 \\
{\bf Example 6.4} (Faster rates under trace constraints)
\begin{itemize}
  \item $\{ \gamma_j(\Sigma)\}_{j = 1}^d$ は$\Sigma$の固有値列で, $\gamma_1(\Sigma)$ がそのうち最大のもの.
  \item $\Sigma$は, 次元に対して独立な定数 $C$ に対し, 次の ``trace constraint'' を満たすとする:
        \[
            \frac{\trace(\Sigma)}{|||\Sigma|||_2}
            = \frac{\sum_{j=1}^d \gamma_j(\Sigma)}{\gamma_1(\Sigma)}
            \le C.
            \tag{6.13}
        \]
  \item $C$は$\Sigma$の(実質的な)rankと見なせる($\because$ (6.13) は$C = \mathrm{rank}(\Sigma)$では常に成立. )
  \item パラメータ $q \in [0,1]$と半径 $R_q >0$ のthe Schatten $q$-``balls''を, 以下で定義する:
        \[
          \bb{B}_q(R_q)
          := \left\{ \Sigma\in S^{d \times d} \Bigg| \sum_{j=1}^d|\gamma_j(\Sigma)|^q \le R_q \right\}.
          \tag{6.14}
        \]
        \begin{itemize}
          \item $q = 0$なら, rank $R_q$ 以下の対称行列の集合.
          \item $q = 1$なら, trace constraintになる.
        \end{itemize}
  \item 任意の非零行列 $\Sigma \in \bb{B}_q(R_q)$ は, (6.13) を $C = R_q / (\gamma_1(\Sigma))^q$ で満たす.
\end{itemize}
\end{frame}

\begin{frame}{}
\begin{itemize}
  \item (6.13) を満たす任意の$\Sigma$に対し, Thm 6.1は高確率で$X$の最大特異値が次のように抑えられることを保証する:
  \[
      \frac{\sigma_{\max}(X)}{\sqrt{n}} \le \gamma_{\max}(\sqrt{\Sigma}) \left(1 + \delta + \sqrt{\frac{C}{n}}\right).
      \tag{6.15}
  \]
  \item $\Sigma = I_d$のときのbound (6.10) と比べると, $C$ が $d$ に置き換わって``実行的なrank''となっている.\endexample
\end{itemize}
\end{frame}

\begin{frame}{}{}
{\bf Proof of Theorem 6.1.}
\end{frame}

\end{document}
