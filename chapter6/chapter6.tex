\documentclass[unicode,aspectratio=169,11pt]{beamer}
\usepackage{amsmath, amssymb, amsthm, color, latexsym, mathrsfs, bm}
\usefonttheme{professionalfonts}
\usepackage{luatexja}
\usepackage[ipaex]{luatexja-preset}
\renewcommand{\kanjifamilydefault}{\gtdefault}

\usetheme[
  sectionpage=none,
  numbering=fraction,
  block=fill
  ]{metropolis}

\def\qed{\hfill $\Box$}
\def\endexample{\hfill $\clubsuit$}
\newcommand{\ex}{\mathbb{E}}
\newcommand{\var}{\mathrm{var}}
\newcommand{\cov}{\mathrm{cov}}
\newcommand{\bb}{\mathbb}
\newcommand{\cc}{\mathcal}
\newcommand{\tr}{\mathrm{T}}
\newcommand{\trace}{\mathrm{tr}}


\title{6. Random matrices and covariance estimation}
\author{担当:みーとみ}
\date{2021年6月30日, 7月7日}


\begin{document}

\maketitle

\begin{frame}{Table of Contents}{}
    \setcounter{tocdepth}{1}
    \tableofcontents
\end{frame}

\section{6.1 Some preliminaries}
\begin{frame}{6.1 Some preliminaries}
  \begin{itemize}
    \item Notationとこの章で使うpreliminary resultsの説明から.
  \end{itemize}
\end{frame}

\subsection{6.1.1 Notation and basic facts}
\begin{frame}{6.1.1 Notation and basic facts}{}
  \begin{itemize}
    \item 行列 $A \in \bb{R}^{n \times m}$ with $n \ge m$ に対し, (順序付き)特異値を
    \[ \sigma_{\max}(A) = \sigma_1(A) \ge \sigma_2(A) \ge \dots \ge \sigma_m(A) = \sigma_{\min}(A) \ge 0 \]
    と書く.
    \item 最小・最大特異値は次のようにcharacterizeされる:
    \[ \sigma_{\max}(A) = \max_{v \in \bb{S}^{m-1}}\| Av \|_2 \quad \mathrm{and} \quad \sigma_{\min}(A) = \min_{v \in \bb{S}^{m-1}}\|Av\|_2, \tag{6.1} \]
    ただし $\bb{S}^{d-1} := \left\{ v \in \bb{R}^d \mid \|v\|_2 = 1 \right\}$ は $\bb{R}^d$ 上のEuclidean unit sphere.
    \item また次の同値性が成り立つ: $||| A |||_2 = \sigma_{\max}(A)$.
  \end{itemize}
\end{frame}

\begin{frame}
  \begin{itemize}
    \item 対称行列の集合を ${\cc{S}}^{d\times d} := \left\{Q\in\bb{R}^{d\times d} \mid Q = Q^\tr\right\}$ とし, その半正定値行列からなる部分集合を
    \[ \cc{S}_+^{d \times d} := \left\{Q \in \cc{S}^{d \times d} \mid Q \succeq 0\right\} \tag{6.2} \]
    と書く.
    \item 任意の対称行列 $Q \in \cc{S}^{d\times d}$ は対角化可能であり, その固有値を
    \[ \gamma_{\max}(Q) = \gamma_1(Q) \ge \gamma_2 \ge \dots \ge \gamma_d(Q) = \gamma_{\min}(Q) \]
    とする.
    \item このとき, $Q \succeq 0 \ \Leftrightarrow \ \gamma_{\min}(Q) \ge 0$.
  \end{itemize}
\end{frame}

\begin{frame}{}{}
  \begin{itemize}
    \item 最小・最大固有値の ``Rayleigh–Ritz variational characterization'':
    \[
        \gamma_{\max}(Q) = \max_{v \in \bb{S}^{d-1}} v^\tr Q v
        \quad \mathrm{and} \quad
        \gamma_{\min}(Q) = \min_{v \in \bb{S}^{d-1}} v^\tr Q v. \tag{6.3}
    \]
    \item 任意の対称行列 $Q$ に対し, その$\ell_2$-operator normは, 
    \[ |||Q|||_2 = \max \left\{ \gamma_{\max}(Q), |\gamma_{\min}(Q)| \right\} = \max_{v \in \bb{S}^{d-1}}\left|v^\tr Qv\right|.\tag{6.4} \]
    \item 最後に, 行列 $A \in \bb{R}^{n\times m}$ with $n \ge m$ に対し, $m$-次元対称行列 $R := A^\tr A$ を考えると, 
    \[ \gamma_j(R) = \left(\sigma_j(A)\right)^2 \quad \mathrm{for}\ j = 1, \dots, m. \]
  \end{itemize}
\end{frame}

\subsection{6.1.2 Set-up of covariance estimation}
\begin{frame}{6.1.2 Set-up of covariance estimation}{}
  \begin{itemize}
    \item $\{x_1,\dots,x_m\}$ は, $\bb{R}^d$上のzero-mean・covariance $\Sigma = \cov(x_1) \in \bb{S}_+^{d\times d}$なる分布からの $n$ 個のi.i.d.サンプルとする.
    \item $\Sigma$ のstandard estimatorは, 次の {\it sample covariance matrix} である:
    \[ \widehat{\Sigma} := \frac{1}{n}\sum_{i=1}^n x_i x_i^\tr.  \tag{6.5} \]
    \item 各 $x_i$ はzero-meanなので $\ex[x_i x_i^\tr] = \Sigma$ であり, $\widehat{\Sigma}$ は $\Sigma$ のunbiased estimator.
    \item したがって $\widehat{\Sigma}-\Sigma$ は期待値ゼロとなり, その$\ell_2$-operator normによって測ったerrorのboundを求めることがこの章のgoalとなる.
  \end{itemize}
\end{frame}

\begin{frame}
  \begin{itemize}
    \item (6.4) の$\ell_2$-operator normの表現より, $|||\widehat{\Sigma}-\Sigma|||_2 \le \epsilon$ は以下と同値:
    \[ \max_{v \in \bb{S}^{d-1}} \left|\frac{1}{n}\sum_{i=1}^n \langle x_i,v_i \rangle^2 - v^\tr \Sigma v\right| \le \epsilon. \tag{6.6}\]
    \item つまり, $|||\widehat{\Sigma} - \Sigma|||_2$ をコントロールすることは, $v$ でindexedされた関数クラス $x \mapsto \langle x, v\rangle^2$ のuniform law of large numbersを示すことと同値になる.
  \end{itemize}
\end{frame}

\begin{frame}{}{}
 \\
  \begin{itemize}
    \item その$\ell_2$-operator normをコントロールすることは, $\widehat{\Sigma}$ の固有値の一様収束も意味する: Weyl's theorem の corollaryより,
    \[ \max_{j = 1, \dots, d} \left|\gamma_j(\widehat{\Sigma}) - \gamma_j\left(\Sigma\right)\right| \le |||\widehat{\Sigma} - \Sigma|||_2. \tag{6.7} \]
     \\
    \item また最後に, ランダム行列 $X \in \bb{R}^{n \times d}$ が, 第$i$行に $x_i^\tr$ を持つものとする
    \[
      X =
      \begin{pmatrix}
        x_1^\tr\\
        \vdots\\
        x_n^\tr
      \end{pmatrix} \in \bb{R}^{n \times d}
    \]
    と,
    \[ \widehat{\Sigma} = \frac{1}{n}\sum_{i=1}^n x_i x_i^\tr = \frac{1}{n}X^\tr X \]
    なので, $\widehat{\Sigma}$ の固有値は $X / \sqrt{n}$ の特異値の2乗となる.
  \end{itemize}
\end{frame}

\section{6.2 Wishart matrices and their behavior}
\begin{frame}{6.2 Wishart matrices and their behavior}
  \begin{itemize}
    \item サンプル $x_i$ は, $d$-次元正規分布 $\mathcal{N}(0, \Sigma)$ から i.i.d. で引かれるとする.
    \item このとき,
    \[
      X =
      \begin{pmatrix}
        x_1^\tr\\
        \vdots\\
        x_n^\tr
      \end{pmatrix} \in \bb{R}^{n \times d}
    \]
    は, {\it $\Sigma$-Gaussian ensemble} から引かれると言う.
    \item Sample covariance $\widehat{\Sigma} = \frac{1}{n}X^\tr X$ は, {a multivariate Wishart distribution} に従う.
  \end{itemize}
\end{frame}

\begin{frame}{}{}
  \begin{block}{Theorem 6.1}
    $X \in \mathbb{R}^{n \times d}$ は $\Sigma$-Gaussian ensembleから引かれるとする.
    このとき, 任意の $\delta > 0$ に対し, 最大特異値 $\sigma_{\max}(X)$ は以下のupper deviation inequalityを満たす:
    \[
      \bb{P} \left[
        \frac{\sigma_{\max}(X)}{\sqrt{n}}
        \ge \gamma_{\max}\left(\sqrt{\Sigma}\right) (1+\delta) + \sqrt{\frac{\trace(\Sigma)}{n}}
      \right]
      \le \exp\left(-\frac{n\delta^2}{2}\right).
      \tag{6.8}
    \]
    さらに$n \ge d$なら, 最小特異値 $\sigma_{\min}(X)$ は以下のlower deviation inequalityを満たす:
    \[
        \bb{P} \left[
        \frac{\sigma_{\min}(X)}{\sqrt{n}}
        \le \gamma_{\min}\left(\sqrt{\Sigma}\right) (1-\delta) - \sqrt{\frac{\trace(\Sigma)}{n}}
      \right]
      \le \exp\left(-\frac{n\delta^2}{2}\right).
      \tag{6.9}
    \]
  \end{block}
\end{frame}

\begin{frame}{}{}
 \\
{\bf Example 6.2} (Operator norm bounds for the standard Gaussian ensemble)
\begin{itemize}
  \item $W \in \bb{R}^{n \times d}$ は各成分が$\cc{N} (0,1)$ i.i.d.で引かれるrandom matrixとする($\Sigma = I_d$).
  \item Thm 6.1より, $n \ge d$ なら, 確率 $1 - 2 \exp\left(-\frac{n\delta^2}{2}\right)$ 以上で
  \[\frac{\sigma_{\max}(W)}{\sqrt{n}} \le 1 + \delta + \sqrt{\frac{d}{n}}
  \quad \mathrm{and} \quad
  \frac{\sigma_{\min}(W)}{\sqrt{n}} \ge 1 - \delta - \sqrt{\frac{d}{n}}
  \tag{6.10}
  \]
  となる.
  \item よって, 同じ確率で
  \[ \Bigg|\Bigg|\Bigg| \frac{1}{n}W^\tr W - I_d \Bigg|\Bigg|\Bigg|_2 \le 2\epsilon + \epsilon^2,
  \quad \mathrm{where} \epsilon = \sqrt{\frac{d}{n}} + \delta.
  \tag{6.11} \]
  \item したがって, $d/n \to 0$ なら, sample covariance $\widehat{\Sigma} = \frac{1}{n}W^\tr W$ はidentiry matrix $I_d$ の一致推定量となる.\endexample
\end{itemize}
\end{frame}

\begin{frame}{}{}
 \\
{\bf Example 6.3} (Gaussian covariance estimation)
\begin{itemize}
  \item $X \in \bb{R}^{n \times d}$ は$\Sigma$-Gaussian ensembleからのrandom matrixとする.
  \item このとき $X = W \sqrt{\Sigma}$ と書ける($W \in \bb{R}^{n\times d}$はstandard Gaussian random matrix)ので,
          \[
              \Bigg|\Bigg|\Bigg| \frac{1}{n} X^\tr X - \Sigma \Bigg|\Bigg|\Bigg|_2
              = \Bigg|\Bigg|\Bigg|\sqrt{\Sigma}\left(\frac{1}{n}W^\tr W - I_d\right) \Bigg|\Bigg|\Bigg|_2
              \le |||\Sigma|||_2 \Bigg|\Bigg|\Bigg| \frac{1}{n} W^\tr W - I_d \Bigg|\Bigg|\Bigg|_2.
          \]
  \item したがって (6.11) より, 任意の $\delta > 0$ に対して確率 $1 - 2\exp\left(-\frac{n \delta^2}{2}\right)$ で
          \[
            \frac{||| \widehat{\Sigma} - \Sigma |||_2}{|||\Sigma|||_2}
            \le 2 \sqrt{\frac{d}{n}} + 2\delta + \left(\sqrt{\frac{d}{n}} + \delta\right)^2.
            \tag{6.12} 
          \]
  \item よって, $||| \widehat{\Sigma} - \Sigma |||_2 / |||\Sigma|||_2$ は $d/n \to 0$ である限り $0$ に収束する.\endexample
\end{itemize}
\end{frame}

\begin{frame}{}{}
 \\
{\bf Example 6.4} (Faster rates under trace constraints)
\begin{itemize}
  \item $\{ \gamma_j(\Sigma)\}_{j = 1}^d$ は$\Sigma$の固有値列で, $\gamma_1(\Sigma)$ がそのうち最大のもの.
  \item $\Sigma$は, 次元に対して独立な定数 $C$ に対し, 次の ``trace constraint'' を満たすとする:
        \[
            \frac{\trace(\Sigma)}{|||\Sigma|||_2}
            = \frac{\sum_{j=1}^d \gamma_j(\Sigma)}{\gamma_1(\Sigma)}
            \le C.
            \tag{6.13}
        \]
  \item $C$は$\Sigma$の(実質的な)rankと見なせる($\because$ (6.13) は$C = \mathrm{rank}(\Sigma)$では常に成立. )
  \item パラメータ $q \in [0,1]$と半径 $R_q >0$ のthe Schatten $q$-``balls''を, 以下で定義する:
        \[
          \bb{B}_q(R_q)
          := \left\{ \Sigma\in S^{d \times d} \Bigg| \sum_{j=1}^d|\gamma_j(\Sigma)|^q \le R_q \right\}.
          \tag{6.14}
        \]
        \begin{itemize}
          \item $q = 0$なら, rank $R_q$ 以下の対称行列の集合.
          \item $q = 1$なら, trace constraintになる.
        \end{itemize}
  \item 任意の非零行列 $\Sigma \in \bb{B}_q(R_q)$ は, (6.13) を $C = R_q / (\gamma_1(\Sigma))^q$ で満たす.
\end{itemize}
\end{frame}

\begin{frame}{}
\begin{itemize}
  \item (6.13) を満たす任意の$\Sigma$に対し, Thm 6.1は高確率で$X$の最大特異値が次のように抑えられることを保証する:
  \[
      \frac{\sigma_{\max}(X)}{\sqrt{n}} \le \gamma_{\max}(\sqrt{\Sigma}) \left(1 + \delta + \sqrt{\frac{C}{n}}\right).
      \tag{6.15}
  \]
  \item $\Sigma = I_d$のときのbound (6.10) と比べると, $C$ が $d$ に置き換わって``実行的なrank''となっている.\endexample
\end{itemize}
\end{frame}

\begin{frame}{}{}
{\bf Proof of Theorem 6.1.}
\begin{itemize}
  \item Notation: $\overline{\sigma}_{\max} = \gamma_{\max}(\sqrt{\Sigma}),\ \overline{\sigma}_{\min} = \gamma_{\min}(\sqrt{\Sigma})$.
  \item 最大/最小特異値のupper/lower boundともに以下の2段階で示す:
  \begin{enumerate}
    \item 高確率で特異値が期待値に近いことをconcentration inequalityから示す(Ch.2)
    \item その期待値のboundの導出にGaussian comparison inequalityを用いる(Ch.5)
  \end{enumerate}
   \\
  \item ここでは最大特異値のupper boundのみを示す.
        (最小特異値のlower boundは大体似た方針で示せるがよりテクニカルなのでAppendix (Section 6.6)にまわす.)
\end{itemize}
\end{frame}

\begin{frame}
  \begin{itemize}
    \item $X = W \sqrt{\Sigma}$ と書ける, ただし $W \in \bb{R}^{n\times d}$ はi.i.d. $\cc{N}(0,1)$ entriesをもつ.
    \item $W \mapsto \frac{\sigma_{\max}(W\sqrt{\Sigma})}{\sqrt{n}}$ を$\bb{R}^{nd}$上の実数値写像とみると, これは $L = \overline{\sigma}_{\max}/\sqrt{n}$でLipschitz w.r.t. Euclidean norm.(cf. Example 2.32)
    \item Gaussian r.v. に対するLipschitz関数のconcentration inequality(Thm 2.26)より,
          \[
            \bb{P}\left[\sigma_{\max}(X) \ge \ex[\sigma_{\max}(X)] + \sqrt{n} \overline{\sigma}_{\max} \delta\right]
            \le \exp\left(-\frac{n \delta^2}{2}\right).
          \]
    \item したがって, あとは以下を示せれば良い:
          \[
            \ex[\sigma_{\max}(X)] \le \sqrt{n} \overline{\sigma}_{\max} + \sqrt{\trace(\Sigma)}. \tag{6.16}
          \]
  \end{itemize}
\end{frame}

\begin{frame}
  \begin{itemize}
    \item $\sigma_{\max}(X) = \max_{v' \in \bb{S}^{d-1}}\|Xv'\|_2$ で, $X = W\sqrt{\Sigma},\ v = \sqrt{\Sigma}v'$とすると次のように書ける:
          \[
            \sigma_{\max}(X) = \max_{v \in \bb{S}^{d-1}(\Sigma^{-1})} \|W v\|_2
            = \max_{u \in \bb{S}^{d-1}} \max_{v \in \bb{S}^{d-1}(\Sigma^{-1})} \underbrace{u^\tr W v}_{Z_{u,v}},
          \]
          ただし $\bb{S}^{d-1}(\Sigma^{-1}) := \{v \in \bb{R}^d \mid \|\Sigma^{-\frac{1}{2}}v\|_2 = 1\}$.
    \item $\{ Z_{u, v} ,\ (u, v) \in \bb{T}\}$ where $\bb{T} := \bb{S}^{d-1}\times \bb{S}^{d-1}(\Sigma^{-1})$はzero-mean Gaussian processとみなせる.\\
     \\
    \item 別のGaussian process $\{Y_{u,v},\ (u,v)\in \bb{T}\}$ で $\ex[(Z_{u,v}-Z_{\tilde{u}\tilde{v}})^2] \le \ex[(Y_{u,v}-Y_{\tilde{u}\tilde{v}})^2]$
          for all $(u,v), (u',v') \in \bb{T}$ となるようなものをconstructすることを考える.
    \item すると Sudakov-Fernique comparison(Thm. 5.27)から以下が言える:
          \[ \ex[\sigma_{\max}(X)] = \ex\left[\max_{(u,v)\in \bb{T}} Z_{u,v}\right] \le \ex\left[\max_{(u,v)\in \bb{T}} Y_{u,v}\right]. \tag{6.17}\]
  \end{itemize}
\end{frame}

\begin{frame}
  \begin{itemize}
    \item $(u, v), (\tilde{u}, \tilde{v}) \in \bb{T}$ をgivenとし, $\|v\|_2 \le \|\tilde{v}\|_2$とする.
    \item まず$Z_{u,v} =u^\tr W v= \langle\langle W, uv^\tr \rangle\rangle$となる, where $\langle\langle A,B\rangle\rangle := \sum_{j=1}^n\sum_{k=1}^dA_{jk}B_{jk}$.
    \item $W$ はi.i.d.$\cc{N}(0,1)$ entriesをもつので,
          \[
            \ex\left[(Z_{u,v} - Z_{\tilde{u}\tilde{v}})^2\right]
            = \ex \left[\langle\langle W, uv^\tr-\tilde{u}\tilde{v}^\tr\rangle\rangle^2\right]
            = |||uv^\tr - \tilde{u}\tilde{v}^\tr |||^2_F.
          \]
    \item Frobenius normを変形すると,
          \begin{align*}
             &|||uv^\tr - \tilde{u}\tilde{v}^\tr |||^2_F\\
            =&|||u(v - \tilde{v})^\tr - (u-\tilde{u})\tilde{v}^\tr |||^2_F \\
            =&|||(u-\tilde{u})\tilde{v}^\tr|||^2_F + |||u(v-\tilde{v})-\tr|||^2_F + 2\langle\langle u(v-\tilde{v})^\tr, (u-\tilde{u})\tilde{v}^\tr\rangle\rangle\\
            \le& \|\tilde{v}\|_2^2\|u-\tilde{u}\|_2^2 + \|u\|_2^2\|v-\tilde{v}\|_2^2 + 2(\|u\|_2^2-\langle u,\tilde{u}\rangle)(\langle v,\tilde{v}\rangle-\|\tilde{v}\|_2^2).
          \end{align*}
    \item ここで, $\|u\|_2 = \|\tilde{u}\|_2 = 1$ より $\|u\|_2^2 - \langle u, \tilde{u}\rangle \ge 0$.
    \item 一方, Cauchy-Schwarzと仮定$\|v\|_2 \le \|\tilde{v}\|_2$より, $|\langle v, \tilde{v}\rangle| \le \|v\|_2 \|\tilde{v}\|_2 \le \|\tilde{v}\|_2^2$.
    \item したがって, 
          \[|||uv^\tr - \tilde{u}\tilde{v}^\tr |||^2_F \le  \|\tilde{v}\|_2^2\|u-\tilde{u}\|_2^2 + \|v-\tilde{v}\|_2^2.\]
  \end{itemize}
\end{frame}

\begin{frame}
  \begin{itemize}
    \item $\bb{S}^{d-1}(\Sigma^{-1})$の定義より, $\|\tilde{v}\|_2 \le \overline{\sigma} = \gamma_{\max}(\sqrt{\Sigma})$ なので,
            \[ \ex[(Z_{u,v} - Z_{\tilde{u}, \tilde{v}})^2] \le \overline{\sigma}^2_{\max} \|u-\tilde{u}\|_2^2 + \|v-\tilde{v}\|_2^2.\]
    \item Gaussian process $Y_{u,v} := \overline{\sigma}_{\max}\langle g, u\rangle + \langle h, v\rangle$ を定義する(ただし $g\in\bb{R}^n, h \in \bb{R}^d$ はstandard Gaussian rv's)と,
            \[ E[(Y_{u,v} - Y_{\tilde{u},\tilde{v}})^2] = \overline{\sigma}^2_{\max}\|u-\tilde{u}\|_2^2 + \|v - \tilde{v}\|_2^2. \]
    \item よって Sudakov-Fernique bound (6.17) より,
    \begin{align*}
      \ex[\sigma_{\max}(X)] \le \ex\left[ \sup_{(u,v) \in \bb{T}} Y_{u,v}\right]
      &= \overline{\sigma}_{\max}\ex\left[\sup_{u \in\bb{S}^{d-1}}\langle g, u \rangle\right] + \ex\left[ \sup_{v \in \bb{S}^{d-1}(\Sigma^{-1})}\langle h, v\rangle \right]\\
      &= \overline{\sigma}_{\max}\ex[\|g\|_2] + \ex[\|\sqrt{\Sigma}h\|_2].
    \end{align*}
    \item Jensen's inequalityから, $\ex[\|g\|_2] \le \sqrt{n}$ ans $\ex[\|\sqrt{\Sigma}h\|_2] \le \sqrt{\ex[h^\tr \Sigma h]} = \sqrt{\trace(\Sigma)}$となり, (6.16)が示された.\qed
  \end{itemize}
\end{frame}

\section{6.3 Covariance matrices from sub-Gaussian ensembles}
\begin{frame}{6.3 Covariance matrices from sub-Gaussian ensembles}{}
  \begin{itemize}
    \item 
  \end{itemize}
\end{frame}

\end{document}
