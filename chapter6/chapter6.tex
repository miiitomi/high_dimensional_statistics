\documentclass[unicode,aspectratio=169,11pt]{beamer}
\usepackage{amsmath, amssymb, amsthm, color, latexsym, mathrsfs, bm}
\usefonttheme{professionalfonts}
\usepackage{luatexja}
\usepackage[ipaex]{luatexja-preset}
\renewcommand{\kanjifamilydefault}{\gtdefault}

\usetheme[
  sectionpage=none,
  numbering=fraction,
  block=fill
  ]{metropolis}

\def\qed{\hfill $\Box$}
\def\endexample{\hfill $\clubsuit$}
\newcommand{\ex}{\mathbb{E}}
\newcommand{\var}{\mathrm{var}}
\newcommand{\bb}{\mathbb}
\newcommand{\cc}{\mathcal}
\newcommand{\tr}{\mathrm{T}}
\newcommand{\trace}{\mathrm{tr}}


\title{6. Random matrices and covariance estimation}
\author{担当:みーとみ}
\date{2021年6月30日, 7月7日}


\begin{document}

\maketitle

\begin{frame}{}{}
    \begin{itemize}
        \item 
    \end{itemize}
\end{frame}

\section{6.2 Wishart matrices and their behavior}
\begin{frame}{6.2 Wishart matrices and their behavior}
  \begin{itemize}
    \item サンプル $x_i$ は, $d$-次元正規分布 $\mathcal{N}(0, \Sigma)$ から i.i.d. で引かれるとする.
    \item このとき,
    \[
      X =
      \begin{pmatrix}
        x_1^\tr\\
        \vdots\\
        x_n^\tr
      \end{pmatrix} \in \bb{R}^{n \times d}
    \]
    は, {\it $\Sigma$-Gaussian ensemble} から引かれると言う.
    \item Sample covariance $\widehat{\Sigma} = \frac{1}{n}X^\tr X$ は, {a multivariate Wishart distribution} に従う.
  \end{itemize}
\end{frame}

\begin{frame}{}{}
  \begin{block}{Theorem 6.1}
    $X \in \mathbb{R}^{n \times d}$ は $\Sigma$-Gaussian ensembleから引かれるとする.
    このとき, 任意の $\delta > 0$ に対し, 最大特異値 $\sigma_{\max}(X)$ は以下のupper deviation inequalityを満たす:
    \[
      \bb{P} \left[
        \frac{\sigma_{\max}(X)}{\sqrt{n}}
        \ge \gamma_{\max}\left(\sqrt{\Sigma}\right) (1+\delta) + \sqrt{\frac{\trace(\Sigma)}{n}}
      \right]
      \le \exp\left(-\frac{n\delta^2}{2}\right).
      \tag{6.8}
    \]
    さらに$n \ge d$なら, 最小特異値 $\sigma_{\min}(X)$ は以下のlower deviation inequalityを満たす:
    \[
        \bb{P} \left[
        \frac{\sigma_{\min}(X)}{\sqrt{n}}
        \le \gamma_{\min}\left(\sqrt{\Sigma}\right) (1-\delta) - \sqrt{\frac{\trace(\Sigma)}{n}}
      \right]
      \le \exp\left(-\frac{n\delta^2}{2}\right).
      \tag{6.9}
    \]
  \end{block}
\end{frame}

\begin{frame}{}{}
 \\
{\bf Example 6.2} (Operator norm bounds for the standard Gaussian ensemble)
\begin{itemize}
  \item $W \in \bb{R}^{n \times d}$ は各成分が$\cc{N} (0,1)$ i.i.d.で引かれるrandom matrixとする($\Sigma = I_d$).
  \item Thm 6.1より, $n \ge d$ なら, 確率 $1 - 2 \exp\left(-\frac{n\delta^2}{2}\right)$ 以上で
  \[\frac{\sigma_{\max}(W)}{\sqrt{n}} \le 1 + \delta + \sqrt{\frac{d}{n}}
  \quad \mathrm{and} \quad
  \frac{\sigma_{\min}(W)}{\sqrt{n}} \ge 1 - \delta - \sqrt{\frac{d}{n}}
  \tag{6.10}
  \]
  となる.
  \item よって, 同じ確率で
  \[ \Bigg|\Bigg|\Bigg| \frac{1}{n}W^\tr W - I_d \Bigg|\Bigg|\Bigg|_2 \le 2\epsilon + \epsilon^2,
  \quad \mathrm{where} \epsilon = \sqrt{\frac{d}{n}} + \delta.
  \tag{6.11} \]
  \item したがって, $d/n \to 0$ なら, sample covariance $\widehat{\Sigma} = \frac{1}{n}W^\tr W$ はidentiry matrix $I_d$ の一致推定量となる.
\end{itemize}
\endexample
\end{frame}

\end{document}
